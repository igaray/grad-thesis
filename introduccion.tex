\chapter{Introducci�n}
\label{chap:introduccion}

\section{Enunciado}
\label{sec:enunciado}

El objetivo de este trabajo es presentar en detalle el sistema de
comunicaci�n de percepciones entre agentes utilizado en el sistema
multi-agente desarrollado en el marco de MAPC 2011\footnote{Multi-
Agent Programming Contest 2011 - http://www.multiagentcontest.org/},
la representaci�n de la base de conocimiento utilizada, las decisiones
de dise�o e implementaci�n tomadas, aspectos exitosos y las
posibilidades de mejora.

En esta instancia de la competencia se hizo �nfasis en lograr un
Sistema Multi-Agente puramente distribuido en el cual cada agente
realiza un proceso de toma de decisiones independiente en lugar de
contar con una inteligencia centralizada que decide cu�l ser� la
acci�n a realizar para cada uno de los agentes. 

Para asegurar que las decisiones de cada agente sean lo mas informadas
posibles, se desarroll� un sistema de sincronizaci�n de la base de
conocimiento que opera en una fase previa al proceso deliberativo de
cada agente, asegurando que todos los agentes sepan lo mismo sobre el
estado actual del mundo.

El objetivo particular de este proyecto es la aplicaci�n de
Argumentaci�n para la implementaci�n de di�logos entre agentes
inmersos en un escenario con objetivos determinados.
Puntualmente, se enfocar� la investigaci�n a la plataforma propuesta
en el Multi-Agent Programming Contest, un juego acad�mico donde
agentes independientes compiten por diferentes objetivos.
Sin embargo, el desarrollo de herramientas para implementar tales
formalismos se encuentra en progreso y a un paso m�s lento.
Adem�s, muchas de las herramientas disponibles carecen de una base
formal y suelen ser simplemente un entorno de desarrollo amigable.

\section{Organizaci�n del trabajo}
\label{sec:organizacion_del_trabajo}

El cap�tulo \ref{chap:definiciones_preliminares} da un conjunto de
definiciones y un marco te�rico sobre el cual se trabaj�, abarcando
los conceptos de agentes, arquitectura BDI, programacion l�gica
rebatible, y el contexto de desarrollo

El cap�tulo \ref{chap:arquitectura} describe la arquitectura general e
interna de cada agente, incluyendo la entrada y salida de datos, las
fases de preprocesamiento de las percepciones, el esquema de
representacion de conocimiento utilizada y el proceso deliberativo
realizado por cada agente.

El cap�tulo \ref{chap:servidor_de_percepciones} describe en detalle el
sistema de sincronizacion de la base de conocimientos de cada agente,
su funcionamiento interno, capacidades y limitaciones.
