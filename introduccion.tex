\chapter{Introducción}
\label{chap:introduccion}

\begin{quote}
\scriptsize{
    \emph{
        If your thesis is utterly vacuous   \\
        Use first-order predicate calculus  \\
            With sufficient formality       \\
            The sheerest banality           \\
        Will be hailed by the critics:      \\
            ``Miraculous''                  \\
        }
    - Henry Kautz
}
\end{quote}

\section{Enunciado}
  \label{sec:enunciado}

  El objetivo de este trabajo es presentar en detalle el sistema de
  comunicación de percepciones entre agentes utilizado en el sistema
  multi-agente desarrollado en el marco de MAPC 2011\footnote{Multi-
  Agent Programming Contest 2011 - http://www.multiagentcontest.org/},
  la representación de la base de conocimiento utilizada, las decisiones
  de diseño e implementación tomadas, aspectos exitosos y las
  posibilidades de mejora.
  
  En esta instancia de la competencia se hizo énfasis en lograr un
  Sistema Multi-Agente puramente distribuido en el cual cada agente
  realiza un proceso de toma de decisiones independiente en lugar de
  contar con una inteligencia centralizada que decide cuál será la
  acción a realizar para cada uno de los agentes. 
  
  Para asegurar que las decisiones de cada agente sean lo mas informadas
  posibles, se desarrolló un sistema de sincronización de la base de
  conocimiento que opera en una fase previa al proceso deliberativo de
  cada agente, asegurando que todos los agentes sepan lo mismo sobre el
  estado actual del mundo.
  
  El objetivo particular de este proyecto es la aplicación de
  Argumentación para la implementación de diálogos entre agentes
  inmersos en un escenario con objetivos determinados.
  Puntualmente, se enfocará la investigación a la plataforma propuesta
  en el Multi-Agent Programming Contest, un juego académico donde
  agentes independientes compiten por diferentes objetivos.
  Sin embargo, el desarrollo de herramientas para implementar tales
  formalismos se encuentra en progreso y a un paso más lento.
  Además, muchas de las herramientas disponibles carecen de una base
  formal y suelen ser simplemente un entorno de desarrollo amigable.

\section{Organización del trabajo}
  \label{sec:organizacion_del_trabajo}
  
  El capítulo \ref{chap:definiciones_preliminares} da un conjunto de
  definiciones y un marco teórico sobre el cual se trabajó, abarcando
  los conceptos de agentes, arquitectura BDI, programacion lógica
  rebatible, y el contexto de desarrollo
  
  El capítulo \ref{chap:arquitectura} describe la arquitectura general e
  interna de cada agente, incluyendo la entrada y salida de datos, las
  fases de preprocesamiento de las percepciones, el esquema de
  representacion de conocimiento utilizada y el proceso deliberativo
  realizado por cada agente.
  
  El capítulo \ref{chap:servidor_de_percepciones} describe en detalle el
  sistema de sincronizacion de la base de conocimientos de cada agente,
  su funcionamiento interno, capacidades y limitaciones.
  
