1\chapter{Arquitectura}
\label{chap:Arquitectura}

 \textbf{
 \begin{tabular}{|l|}
 \hline
 DRAFT                                                              \\
 \hline
 \end{tabular}
 }

 Este capítulo describe la arquitectura general del sistema, el 
 protocolo de comunicación con el servidor MASSIM, la 
 arquitectura interna de cada agente, incluyendo la entrada y
 salida de datos, las fases de preprocesamiento de las percepciones,
 el esquema de representacion de conocimiento utilizada y el proceso 
 deliberativo realizado por cada agente.
 
%--------------------------------------------------------ARQ SISTEMA-%
\section{Arquitectura del sistema}
\label{sec:arquitectura_sistema}

 El sistema \texttt{\textbf{d3lp0r}} consiste del conjunto de procesos 
 Agentes y el proceso Servidor de Percepciones. 
 Los Agentes tienen como componentes principales al módulo principal, 
 que dirije la lógica de control del agente, los módulos de 
 comunicación con el servidor MASSim y el Servidor de Percepciones, 
 y los módulos de establecimiento de creencias y deliberacion.
 El Servidor de Percepciones esta implementado en un único módulo, y 
 tiene como componentes principales la lógica de control y el protocolo 
 de comunicación. 

 % ORIGIN: report
 El programa principal del agente esta implementado en Python y maneja
 toda comunicación con los servidores, parseo de los mensajes XML, 
 procesamiento de la información contenida en las percepciones para 
 transformarlas a un formato adecuado para su aserción en la base de 
 conocimiento del agente, y la generación del XML que representa las 
 acciones que toma el agente y es enviado al servidor MASSim. 

 % ORIGIN: report
 La inicialización del agente consiste en la apertura de la conexión al 
 servidor MASSIm y la subsiguiente autentificación, la apertura de la 
 conexión al Servidor de Percepciones, y la inicialización del motor 
 Prolog.
 Luego de la fase de inicialización se ingresa al bucle principal, en 
 el cual se reciben y parsean mensajes del servidor MASSim y se responde
 a ellos de manera adecuada.  

 Al recibir un mensaje de tipo \texttt{sim-start}, la información 
 presente en el mensaje tal como el rol del agente y los parametros
 de la simulación se asertar en la base de conocimiento del agente y 
 se inicia el ciclo de percibir-actuar.

 Cada iteracion del bucle de percibir-actuar espera un mensaje de tipo
 \texttt{request-action} desde el servidor MASSim y parsea el XML para
 transformarlo en un diccionarion (arreglo asociativo) de Python. 
 Los elementos de la percepción se separan en una sección ``publica'', 
 la cual es enviada al Servidor de Perceciones, y otra ``privada''.

 A continuación el agente enviará la sección pública de su percepción 
 al Servidor de Percepciones y espererá la ``percepción global'' que 
 contendrá el resto de la información percibida por el equipo. 
 La percepción global se unirá con su propia, y será asertada en su 
 b
 ase de conocimiento, estableciendo sus creencias. 

 El módulo de decisión implementado en Prolog es consultado para 
 determinar la siguiente acción que ejecutará el agente. 
 Una vez que el flujo de control retorna al programa Python con el 
 resultado de la fase deliberativa, el mensaje XML correspondiente es 
 generado y enviado al servidor MASSim.   

%---------------------------------------------------------ARQ AGENTE-%
\section{Arquitectura del agente}
 \label{sec:arquitectura_agente}

 Esta sección tiene el objetivo de analizar el diseño y estructura de 
 un agente individual.

%-------------------------------------------------------------DISEÑO-%
\subsection{Diseño general}
 \label{sub:diseno_general}
 
 % ORIGEN: marcov 
 El programa agente presenta una estructura simple en cuanto a su
 división.
 La interacción con el entorno y el procesamiento inicial de la
 información recibida finalizan con la generación de una serie de
 creencias que son incorporadas a la base de conocimiento mantenida por
 el agente.
 Este conjunto de creencias es empleado posteriormente por el módulo
 encargado de tomar decisiones.
 La forma en que se estructuran los componentes principales es
 detallada a continuación.

 % ORIGEN: report
 El agente parsea las opciones pasadas por linea de comando al iniciar 
 que determinaran sus paremetros de comportamiento.
 Entre ellos estan los detalles de autentificación que utilizará con 
 el MASSim server (nombre de usuario y contraseña), si esta habilitado 
 el registro de eventos y su nivel de verbosidad, la ubicación en la 
 red del Servidor de Percepciones, y si el agente operará en modo 
 ``dummy'' (en el cual no utiliza argumentación en el proceso 
 deliberativo) o no. 

 % ORIGEN: report
 El mensaje de inicio de simulación es enviado por el MASSim server 
 cada vez que un agente se conecta.  
 
 Los posibles tipos de mensajes son \texttt{sim-start}, \texttt{sim-end}, \texttt{request-action} y \texttt{bye}. 

 Para cada solicitud de acción, la cual incluye la percepción del 
 agente para ese turno, el agente procesará el mensaje, enviando la 
 información que desea compartir con los demás agentes del equipo al 
 servidor de Percepciones, reincorporá la respuesta del 
 Servidor de Percepciones a su base de conocimiento, iniciará el 
 proceso deliberativo para llegar a una decisión sobre cual acción 
 tomar, y finalmente enviará un mensaje representando su decisión al 
 servidor MASSim.   

\subsubsection{Estructura básica del agente}
 \label{subsub:estructura_basica_de_agente}
  
 El programa principal del agente es el encargado de manejar la
 comunicación con los servidores, tanto el del juego como el de
 percepciones (presentado a continuación).
 También es responsable de parsear y procesar la información contenida
 en la percepción para darle el formato interpretado por la base de
 conocimientos, y enviar la acción que ha sido elegida por el módulo
 de toma de decisiones.
 
 \begin{figure}
 \centering
 \includegraphics[scale=.4]{graficos/eps/agent_architecture.eps}
 \caption{Diagrama de la arquitectura del agente. Las líneas punteadas
 representan el flujo de control, y las líneas contínuas representan el
 flujo de datos.}
 \label{fig:architecture}
 \end{figure}
 
 % El servidor de percepciones (SP) es un programa independiente,
 % encargado de unificar las percepciones de todos los agentes que se
 % encuentran en ejecución.
 % Recibe sus percepciones individuales y retorna a cada uno de ellos el
 % conjunto de datos que aún no poseen, de manera que todos los agentes
 % del equipo cuenten con la misma información en cuanto al estado del
 % escenario.
  
 % En cada iteración de la simulación, el agente recibe un mensaje por
 % parte del servidor del juego, el cual contiene la información
 % asociada a la percepción del turno en disputa.
 % Este mensaje es parseado y traducido en una estructura que permite
 % manipular los datos con mayor facilidad.
 % Los datos son divididos en dos conjuntos, uno ``público'', el cual es
 % compartido con los demás agentes del equipo, y uno ``privado''.
 % La sección pública de datos es compartida a través del mencionado
 % servidor de percepciones.
 
 % El agente une entonces su propia percepción con la percepción global
 % recibida del servidor de percepciones, y genera un único conjunto de
 % datos.
 % Esta información es incorporada a la base de conocimientos,
 % estableciendo nuevas creencias para el agente.
 
 % El módulo de toma de decisiones, analizado en la sección
 % \ref{sec:arquitectura_bdi}, es el que implementa el modelo BDI
 % respetado por el agente.
 % Este módulo es consultado en cada iteración para obtener la próxima
 % acción a ser ejecutada.
 % Una vez que el flujo de control retorna al programa principal, la
 % acción seleccionada es enviada al servidor del juego.
  
\subsubsection{Base de conocimiento}
 \label{subsub:base_de_conocimiento}
  
 % Como fue mencionado, la percepción del agente en cada iteración es
 % convertida a una estructura de datos que permite, de manera más
 % sencilla, manipular y compartir la información.
 % Cuando el agente cuenta con todos los datos relativos a las
 % percepciones del equipo, la base de conocimiento puede ser
 % actualizada convenientemente.
 % Una colección de predicados de Prolog consultados desde el programa
 % principal se encarga de verificar que la información existente no
 % resulte sobreescrita, y que información redundante no sea incorporada.
 
 % La información que constituye conocimiento certero sobre el estado del
 % escenario es almacenada mediante términos, que sirven como parámetros
 % del predicado \texttt{k/1} (\textit{knowledge}).
 % Cada uno de los datos de interés es representado mediante un término
 % diferente.
 % En muchos casos, esta clase de términos incluyen un parámetro ligado
 % al número de turno en el cual el dato fue percibido.
 % De esta forma, es posible realizar ciertos análisis, como por
 % ejemplo, considerar obsoleta la información de una determinada
 % antigüedad.
 
 % \begin{verbatim}  
 %   k(equipoAgente(Agente, Equipo)).
 %   k(valorNodo(Nodo, Valor)).
 %   k(arco(Nodo1, Nodo2, Costo)).
 %   k(posicionAgente(Agente, Turno, Posición)).
 %   k(equipoNodo(Turno, Nodo, Dueño)). 
 % \end{verbatim}
 
 % Las creencias que provienen de inferencias y cálculos realizados a
 % partir de información ya existente también son almacenadas mediante
 % términos, en este caso parámetros %argumentos del predicado
 % \texttt{b/1} (\textit{beliefs}). 
 
 % Este tipo de creencias es empleado directamente por el módulo
 % encargado de la toma de decisiones, y se mantienen vigentes sólo
 % durante el turno en el cual fueron generadas. Es decir, que, al
 % finalizar cada turno, son descartadas para evitar futuros problemas o
 % inconcistencias.
 
 % \begin{verbatim}
 %   b(estoyEnLaFrontera).
 %   b(posibleExplorar(Nodo)).
 %   b(haySaboteador(Nodo)).
 % \end{verbatim}
 
 % Existe cierta información que es formulada de manera hipotética.
 % Se trata de datos surgidos de suposiciones realizadas sobre posibles
 % estados futuros del escenario, a partir de su estado actual.
 % Este tipo de datos resulta fundamental para facilitar los cálculos
 % realizados por los algoritmos que se encargan de buscar formas de
 % maximizar el puntaje del equipo.
 % Dado que no constituye información real, sino posible a futuro, se
 % almacena mediante parámetros de un predicado especial, \texttt{h/1}
 % (\textit{hypothetical}).
  
 % \begin{verbatim}
 %   h(nodoEquipo(Nodo, Dueño)).
 %   h(posicion(Turno, Agente, Nodo)). 
 % \end{verbatim}
 
 % Las intenciones surgen del proceso argumentativo explicado más
 % adelante, y son representadas utilizando términos.
 % Si la intención no posee argumentos, entonces es representada mediante
 % un átomo.
 % En otro caso, se emplea un functor que denota el nombre de la
 % intención, acompañado por un argumento.
 % Las acciones, por el contrario, son representadas a través de listas.
 % El primer elemento de la lista es un atómo denotando el tipo de
 % acción.
 % El resto de la lista contiene, ocasionalmente, un término que indica
 % el argumento de la acción, como por ejemplo el nombre de un nodo o un
 % agente.
 % Los planes son representados mediante listas de acciones, es decir,
 % listas de listas.
 
 % Contrario a lo que ocurre con las creencias, tanto las intenciones
 % como los planes constituyen información que debe perdurar en la base
 % de conocimiento tantos turnos como sea necesario.
 % Para este tipo de datos se emplean hechos específicos que cuentan con
 % un único argumento.
 
 % \begin{verbatim}
 %   intencion(explorar(vertex7)).
 %   plan([[recharge], [goto, vertex7], [survey]]).
 % \end{verbatim}
  
%----------------------------------------------------------------BDI-%
\subsection{Arquitectura BDI} 
 \label{sub:arquitectura_bdi}

 % El módulo de toma de decisiones es consultado por el programa
 % principal, obtiene la próxima acción a ser ejecutada, y la retorna
 % para que pueda ser enviada.
 % Esta es una secuencia que se reitera en cada uno de los turnos de la
 % simulación, con una característica: cuando es necesario plantear y
 % planificar una nueva meta, intervienen una serie de componentes
 % especiales, que difieren de aquellos involucrados cuando se cuenta con
 % una meta ya planificada.
 % Cada uno de estos componentes es descrito en esta sección.
 
 % \begin{figure}[ht]
 % \centering
 % \includegraphics[scale=.3]{graficos/eps/agent_prolog.eps}
 % \caption{Diagrama de la arquitectura interna del agente,
 % particularmente todo lo relacionado con la toma de decisiones, hecha
 % en Prolog.
 % Las líneas punteadas representan el flujo de control, y
 % las líneas contínuas representan el flujo de datos.}
 % \label{fig:agent_prolog}
 % \end{figure}
  
\subsubsection{Seteo de creencias}
 \label{subsub:seteo_de_creencias}
  
 % El seteo de creencias es llevado a cabo cada vez que el agente se
 % dispone a seleccionar una nueva intención.
 % Incluye la generación de aquellos datos que pueden permitir al agente
 % realizar una elección lo más acertada posible.
 % Se trata de inferencias realizadas en base al estado del escenario, es
 % decir, aquella información que, como fue mencionado, es almacenada en
 % \texttt{b/1}.
 % No forma parte de este proceso la información proveniente de la
 % percepción, ya que el estado del entorno es actualizado en cada turno
 % de manera previa.
 % Como se detalla a continuación, distintos tipos de creencias pueden
 % pueden estar relacionadas a distintos factores, como el rol del
 % agente, su estado, o los deseos en análisis.
  
 % \paragraph{Creencias generales}
  
 % Existe un conjunto de creencias que resultan de utilidad general para
 % todo el proceso de decisión.
 % Por esta razón, son las primeras en ser calculadas y almacenadas
 % durante el seteo de creencias.
 % Entre los datos incluidos, se encuentra el puntaje que están aportando
 % las zonas armadas, la diferencia de puntos que puede producirse si el
 % agente abandona su posición (ambos puntajes calculados utilizando una
 % versión propia y optimizada del \textit{algoritmo de coloreo} de la
 % competencia, que a su vez será reutilizado en la etapa de resolución
 % de conflictos), y la seguridad que brindan las distintas ubicaciones
 % posibles en cuanto a la presencia de agentes saboteadores enemigos.
  
 % \paragraph{Deseos}
  
 % Como se detallará en la sección siguiente, el proceso de toma de
 % decisión conlleva el pesaje de todos los posibles deseos del agente, y
 % la posterior selección del más beneficioso.
 % Dichos deseos surgen de un conjunto predefinido, y pueden, según sea
 % el caso, estar instanciados con diferentes entidades del juego, como
 % agentes o nodos. Para que esta selección sea posible, es necesario
 % determinar, de manera previa, qué deseos e instanciaciones son
 % realmente factibles, y por lo tanto deben ser tenidos en cuenta, y
 % cuales pueden ser descartados anticipadamente.
 % Para esto se analizan distintas condiciones como, por ejemplo, la
 % distancia a un nodo que no ha sido explorado.
 % Si el nodo se encuentra a una distancia que supera una cota pre-
 % establecida, entonces el deseo de explorar ese nodo no es contemplado.
 % Los deseos e instanciaciones considerados factibles son seteados en la
 % base de conocimiento.
 
 % \begin{verbatim}
 %   b(posibleExplorar(vertex4)).
 % \end{verbatim}
  
 % \paragraph{Creencias específicas} 

 %% Seteo de beliefs para cada deseo.
  
 % Junto con los deseos a ser evaluados, es necesario incluir en la base
 % de conocimiento un conjunto de creencias relacionadas a estos deseos.
 % Entre las más importantes, se encuentran las distancias que existen
 % desde la posición actual del agente a los distintos nodos de interés,
 % y la diferencia de puntaje que se produce en caso que el agente se
 % desplace a dichas ubicaciones.
 % Estos datos resultan fundamentales, ya que afectan directamente la
 % valuación que se realiza de cada deseo, y por lo tanto la posterior
 % selección.
  
 % En esta etapa, también se produce el seteo de datos requeridos
 % posteriormente, como son los caminos a los diferentes nodos
 % analizados.
 % Los algoritmos empleados para la búsqueda de caminos almacenan todos
 % los caminos hallados, en forma de secuencia de acciones, de manera que
 % la etapa de planificación, ejecutada cuando se ha decidido una
 % intención, pueda ser realizada en forma simple e directa.
  
 % \paragraph{Creencias especiales} 

 %% Seteo de beliefs en caso de agente deshabilitado.
  
 % Cuando el agente se encuentra en una situación de peligro, esto es, no
 % posee el rol de saboteador y hay un saboteador enemigo en su posición,
 % o fue atacado en el turno anterior, el conjunto de creencias seteadas
 % se reduce.
 % En estos casos, sólo son tenidos en cuenta los nodos vecinos, dado que
 % representan las vías de escape más rápidas; son calculadas las
 % distancias a estos (en cantidad de turnos), y las diferencias de
 % puntaje que produciría el desplazamiento del agente.
 % Esto tiene el objetivo de minimizar la cantidad de deseos
 % considerados: sólo son evaluadas la posibilidad de permanecer en la
 % misma ubicación (si el beneficio en puntaje es considerable), y las
 % distintas alternativas de defensa propia que pueden llevar al agente a
 % superar el peligro.
  
%------------------------------------------------------ARGUMENTACION-%
\subsection{Argumentación}
 \label{sub:argumentacion}
  
 % Una vez finalizado el seteo de creencias, el agente procede a la
 % selección de la próxima intención.
 % Para esto, se toma cada uno de los deseos marcados como factibles en
 % la base de conocimiento, y se los evalua junto a una serie de
 % ``condiciones'' particulares.
 % Se considera que existen razones para creer realizables sólo aquellos
 % deseos que satisfacen sus condiciones.
 % Para estos, se obtiene un valor que representa su peso, en términos
 % del beneficio que conllevan para el equipo.
 % El deseo que presenta el mayor peso entre los analizados, se convierte
 % en la nueva meta del agente, la cual es almacenada hasta ser alcanzada
 % o reemplazada.
 
 % Tanto la evaluación como el pesaje de los deseos, son llevados a cabo
 % empleando \textit{argumentación} en un módulo especial, implementado
 % con la ayuda de \textit{DeLP}.
  
%------------------------------------------------------PLANIFICACION-%
\subsection{Planificación}
 \label{sub:planificacion}
  
 % La planificación consiste en obtener la secuencia de acciones que
 % llevan al cumplimiento de la intención propuesta.
 % Esta lista está compuesta por las acciones que le permiten al agente
 % posicionarse en el nodo deseado, y, en algunos casos, una acción
 % concreta a realizar. Como se dijo anteriormente, en la etapa de seteo
 % de creencias, todos los caminos hallados por el algoritmo de búsqueda
 % son almacenados. Dicho algoritmo fue implementado de manera tal que
 % los caminos no están constituidos por nodos o vértices, sino por una
 % secuencia optimal de acciones, que tiene en cuenta no sólo el nodo
 % destino, sino también los recursos del agente, y la meta final a
 % realizar (en caso de haber una acción final).
 % De esta forma, cualquiera haya sido la intención elegida, el agente
 % cuenta en su base de conocimiento con el plan necesario para
 % cumplirla.
 % La planificación se resume entonces a tomar las acciones
 % correspondientes, y establecerlas efectivamente como el plan a seguir.
  
 % Alternativamente, esta etapa puede introducir ciertas acciones con el
 % objetivo de optimizar el uso del turno.
 % En aquellas situaciones en que el agente se dispone a permanecer
 % inactivo, la acción nula (\texttt{skip}) puede ser reemplazada por la
 % acción de recargar energía, si es que esta resulta más productiva.
  
%----------------------------------------------------------EJECUCION-%
\subsection{Ejecución}
 \label{sub:ejecucion}
  
 % Dado que el plan se encuentra almacenado de manera completa y
 % ordenada, la ejecución se realiza en forma directa.
 % Se toma la próxima acción, es decir, la primera acción del plan
 % restante, y se la retorna al módulo principal del programa.
 % Éste se encarga posteriormente de enviarla al entorno, para que se
 % convierta finalmente en la siguiente acción realizada por el agente.
  
\subsubsection{Condición de corte}
 \label{subsub:condicion_de_corte}
  
 % Existen situaciones en las que el paso de los turnos genera que el
 % cumplimiento de una meta se vuelva inalcanzable, innecesario,
 % riesgoso, o menos productivo de lo previsto, por lo que resulta más
 % beneficioso abortar el plan existente, y seleccionar una nueva
 % intención.
 % Ésta es una etapa de verificación, que tiene como objetivo la
 % detección de este tipo de situaciones.
 % Es ejecutada sólo en aquellos turnos en los que el agente se encuentra
 % siguiendo el plan de una intención previamente determinada.
 
 % Cada deseo o esquema de deseo cuenta con una serie de
 % \textbf{condiciones de corte}, que son evaluadas al inicio de cada
 % turno, en caso de existir un plan establecido.
 % Si se verifica que alguna de estas condiciones se satisface, entonces
 % la intención es descartada, y el agente ingresa en un nuevo proceso de
 % selección. Entre las condiciones de corte tenidas en cuenta, se
 % encuentran:
 
 % \begin{itemize}
 % \item Que haya pasado una determinada cantidad de turnos desde el
 % inicio del plan.
 
 % \item Que el agente se encuentre deshabilitado.
 
 % \item Que haya sido atacado o se encuentre amenazado por un enemigo.
 
 % \item Que la meta haya sido alcanzada por un compañero de equipo.
 % \end{itemize}
  
\subsubsection{Re-planificación}
 \label{subsub:replanificacion}
  
 % La fase de re-planificación consiste en elaborar nuevamente el plan
 % que permite alcanzar la meta propuesta, sin modificar dicha meta. Este
 % paso, como el anterior, se realiza en los turnos en los que el agente
 % posee un plan pre-calculado.
 % Dado que en estos turnos no es necesaria la obtención de una nueva
 % intención, proceso que implica el mayor insumo de tiempo, la inclusión
 % de la re-planificación no afecta el funcionamiento normal del agente,
 % en términos de tiempo de ejecución.
 
 % Por el contrario, existe una mejora en el desempeño del equipo,
 % surgida de un mejor aprovechamiento de la información percibida.
 % Los agentes actualizan su información sobre el estado del mundo en
 % cada turno.
 % Datos como el estado en que se hallan los recursos del agente, la
 % incorporación de nodos y arcos hasta el momento desconocidos, o las
 % nuevas ubicaciones de los otros agentes, permiten elaborar planes más
 % precisos y ajustados a la realidad que los originalmente diseñados.
 % Así, los agentes son capaces de cumplir sus metas con mayor facilidad,
 % o abortarlas si es necesario.











%---------------------------------------------CONEXION MASSIM-SERVER-%
\section[Conexión con el MASSim Server]
 {Conexión del Agente con el Servidor MASSim}
 \label{sec:conexion_massim}

 El protocolo de comunicación con el servidor MASSim especificado por 
 el enunciado del concurso MAPC define que los agentes y el servidor 
 intercambian mensajes en formato XML codificados por UTF-8, con un 
 byte nulo para indicar el final del mensaje. 

%--------------------------------------------PROTOCOLO MASSIM-SERVER-%
\subsection[Protocolo del MASSim Server]
 {Protocolo de comunicación con el servidor MASSim}
 \label{sub:protocolo_massim}

 TODO

%---------------------------------------------------FORMATO MENSAJES-%
\subsection{Formato de mensajes}
 \label{sub:formato_mensajes}

 TODO

\subsubsection{Percepciones}
 \label{subsub:percepciones}

 TODO

\subsubsection{Acciones}
 \label{subsub:acciones}

  Las acciones se representan con la cadena de caracteres:
  
  \begin{verbatim}
  <?xml version="1.0" encoding="UTF-8" standalone="no"?>
    <message type="action">
      <action id=\"ID\" type=\"TYPE\"/>
    </message>\0
  \end{verbatim}
  
  En el caso de las acciones que requieren un parámetro adicional, la
  representación es:
  
  \begin{verbatim}
  <?xml version="1.0" encoding="UTF-8" standalone="no"?>
    <message type="action">
      <action id="ID" param="PARAM" type="TYPE"/>
    </message>\0
  \end{verbatim}
  
  donde {\tt ID} es el identificador de mensaje enviado por el servidor
  MASSim en la percepcion, {\tt TYPE} es el tipo de accion y {\tt PARAM}
  es el parametro. 
  
  {\tt TYPE} puede ser uno de:
  
  \begin{itemize}
  \item \tt{skip}
  \item \tt{goto}
  \item \tt{attack}
  \item \tt{parry}
  \item \tt{probe}
  \item \tt{survey}
  \item \tt{inspect}
  \item \tt{repair}
  \item \tt{recharge}
  \item \tt{buy}
  \end{itemize}

 TODO

