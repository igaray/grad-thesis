% Contexto de la tesis (background formal, y contexto del desarrollo

\chapter{Definiciones preliminares} 
\label{chap:definiciones_preliminares}

En este cap�tulo se revisar�n algunas definiciones de conceptos
t�cnicos, para posteriormente utilizarlos sin ambig�edad durante el
resto de la presentaci�n.

\section{Agente inteligente}
\label{sec:agente_inteligente}

Un agente es una entidad computacional aut�noma, que puede percibir su
entorno a trav�s de sensores, y actuar en dicho entorno utilizando
efectores.
Usualmente, la informaci�n que un agente percibe de su entorno es s�lo
parcial.
Los agentes toman decisiones a partir de la informaci�n contenida en
su base de conocimiento, siguiendo diferentes conjuntos de reglas
propuestas, y act�an de manera acorde a la decisi�n tomada.
Dichas acciones, a su vez, pueden producir efectos en el entorno.

Actualmente los agentes tienen un campo de aplicaci�n muy amplio y
existen muchos tipos de agentes diferentes (por ejemplo:
\textit{reactivos}, \textit{deliberativos}, \textit{inteligentes},
\textit{de interface}, \textit{colaborativos}), los cuales a su vez
est�n orientados a distintos entornos de aplicaci�n.

En la mayor�a de los casos, los agentes no existen por s� solos, sino
que participan de un Sistema Multi-Agente (SMA).

\section{Sistema Multi-Agente}
\label{sec:sistema_multiagente}

En un Sistema Multi-Agente (SMA) mas de un agente interact�an para
lograr un objetivo o realizar una tarea com�n.
Cada agente tiene informaci�n incompleta y capacidades limitadas, el
control del sistema es distribuido, los datos est�n descentralizados,
y la computaci�n es asincr�nica.
Los agentes se desenvuelven en un entorno din�mico y cambiante, el
cual no puede predecirse y se ve afectado por las acciones que son
llevadas a cabo.

Un aspecto importante en SMA es la comunicaci�n entre agentes, la cual
puede ser necesaria para que los agentes compitan o cooperen de
acuerdo a sus metas individuales. 
Los di�logos con otros agentes del mismo ambiente son, actualmente, un
�rea de estudio intensivo.
