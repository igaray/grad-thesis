% Contexto de la tesis (background formal, y contexto del desarrollo
\chapter{Definiciones preliminares} 
\label{chap:definiciones_preliminares}

En este cap�tulo se revisar�n algunas definiciones de conceptos
t�cnicos, para posteriormente utilizarlos sin ambig�edad durante el
resto de la presentaci�n.

\section{Agente inteligente}
\label{sec:agente_inteligente}

Un agente es una entidad computacional aut�noma, que puede percibir su
entorno a trav�s de sensores, y actuar en dicho entorno utilizando
efectores. Usualmente, la informaci�n que un agente percibe de su
entorno es s�lo parcial. Los agentes toman decisiones a partir de la
informaci�n contenida en su base de conocimiento, siguiendo diferentes
conjuntos de reglas propuestas, y act�an de manera acorde a la
decisi�n tomada. Dichas acciones, a su vez, pueden producir efectos en
el entorno.

Actualmente los agentes tienen un campo de aplicaci�n muy amplio y
existen muchos tipos de agentes diferentes (por ejemplo:
\textit{reactivos}, \textit{deliberativos}, \textit{inteligentes},
\textit{de interface}, \textit{colaborativos}), los cuales a su vez
est�n orientados a distintos entornos de aplicaci�n.

En la mayor�a de los casos, los agentes no existen por s� solos, sino
que participan de un Sistema Multi-Agente (SMA).


\section{Sistema Multi-Agente}
\label{sec:sistema_multiagente}

Es un sistema en el cual muchos agentes interact�an para conseguir
alg�n objetivo o realizar alguna tarea com�n. En los sistemas multi-
agente, cada agente tiene informaci�n incompleta y capacidades
limitadas, el control del sistema es distribuido, los datos est�n
descentralizados, y la computaci�n es asincr�nica. Los agentes se
desenvuelven en un entorno din�mico y cambiante, el cual no puede
predecirse y se ve afectado por las acciones que son llevadas a cabo.

Un aspecto importante en SMA es la comunicaci�n entre agentes, la cual
puede ser necesaria para que los agentes compitan o cooperen de
acuerdo a sus metas individuales. Los di�logos con otros agentes del
mismo ambiente son, actualmente, un �rea de estudio intensivo.


\section{Modelo BDI}
 \label{sec:modelo_bdi}
 
 % El \textit{modelo Creencia-Deseo-Intención}, en adelante \textit{BDI}
 % (\textit{Belief-Desire-Intention}), es un modelo desarrollado para el
 % diseño de agentes inteligentes, basado en una vista simplificada de la
 % inteligencia humana.
 % Como se analizará en la sección \ref{sec:arquitectura_bdi}, el sistema
 % presentado en este trabajo implementa una  adaptación de dicho modelo.
 % Por esta razón, se introducen en esta sección los  conceptos básicos
 % relacionados, que sirvieron de base para nuestro desarrollo.
 
 
 % El modelo BDI está dedicado al modelado formal del razonamiento
 % práctico, es  decir, la formalización de las bases y explicaciones
 % psicológicas y filosóficas  (provenientes principalmente de la
 % filosofía de la mente y de la acción) de los  conceptos de agente,
 % acción, intención, creencia, voluntad, deliberación,  razonamiento de
 % medios y fines, etc.
 % El razonamiento práctico es incorporado  en agentes (por ejemplo, los
 % seres humanos) capaces de perseguir y, por lo tanto,  comprometerse
 % con una determinada meta factible (una acción en particular)  a través
 % de una cuidadosa planificación de los medios, de las condiciones
 % previas  y las acciones que conducen a ese objetivo.


 % Estos conceptos son incorporados al modelo mediante la implementación
 % de los  aspectos principales de la teoría del razonamiento práctico
 % humano de Michael Bratman (también referido como Belief-Desire-
 % Intention, o BDI).
 % Es decir, implementa  las nociones de creencia, deseo y (en
 % particular) intención, de una manera inspirada  por Bratman.
 % Una discusión más extensa puede ser encontrada en el mencionado
 % trabajo de Bratman\cite{brat99} y Searle\cite{searle1985}.
 
 
 % Este basamento teórico permite al modelo resolver un problema
 % particular que  se presenta en la programación de agentes.
 % Provee un mecanismo para separar la  actividad de seleccionar un plan
 % de la ejecución de los planes actualmente activos.
 % Los agentes BDI son capaces de balancear el tiempo invertido en
 % deliberar sobre los planes (elegir qué hacer) y ejecutar estos planes
 % (llevarlo a cabo).
 % La actividad  de crear los planes en primera instancia, escapa al
 % alcance del modelo.

\subsection{Creencias, Deseos e Intenciones}
 \label{sec:creencia_deseos_intenciones}
 
 % Las \textit{creencias, deseos e intenciones} son consideradas estados
 % mentales  intencionales (de forma opuesta a, por ejemplo, el dolor o
 % el placer).
 % Las \textit{creencias}  describen la percepción de la realidad a
 % través de datos provenientes de  los sentidos.
 % Representan el estado \textit{informacional} del agente; comprenden el
 % conocimiento (tanto de sentido común como teórico) sobre el mundo, ya
 % sea  externo o interno.
 % Están sujetas a revisión, lo que implica que pueden  cambiar en el
 % futuro, pueden ser rechazadas o agregadas.
 
 
 % Los \textit{deseos} e \textit{intenciones}, pueden ser vistos como
 % conceptos que  se asemejan, aunque con algunas sutiles diferencias.
 % Los deseos representan el  estado \textit{motivacional} del agente;
 % consisten en su voluntad de alcanzar  ciertos objetivos o situaciones.
 % Entre los deseos, se distingue la noción de  \textit{meta}.
 % Una meta es un deseo que ha sido adoptado por el agente para  ser
 % perseguido activamente.
 % Esta definición impone la restricción de que el  conjunto de metas, o
 % deseos activos, debe ser consistente.
 
 
 % Por último, el concepto de intención representa el estado
 % \textit{deliberativo} del agente, lo que el agente ha elegido hacer.
 % Constituyen deseos para los cuales el agente se ha comprometido.
 % Es una noción más ligado al compromiso que es  asumido, en función
 % alcanzar los estados o situaciones deseadas.

\subsection{Deliberación y planificación}
 \label{sec:deliberacion_planificacion}
 
 % Por \textit{deliberación} entendemos lo que la literatura denomina
 % \textit{silogismo práctico}, es decir, la inferencia de una intención
 % a partir de un conjunto de creencias y deseos.
 % Esto es, la selección de un deseo factible.
 % Una \textit{decisión}  consiste en el último paso de este proceso de
 % inferencia mediante el cual  resulta electo uno de muchos deseos y
 % potenciales intenciones.
 % Es, por esto, un concepto ligado directamente al de intención.
 % Definir una intención implica, en  términos de agentes implementados,
 % comenzar la ejecución de un \textit{plan}.
 
 % Una \textit{acción} puede ser definida, intuitivamente, como la
 % ejecución de una operación que causa un determinado efecto o
 % consecuencia sobre el entorno en el cual se está desempeñando el
 % agente.
 % La \textit{planificación} consiste  en la disposición de una secuencia
 % de acciones con el fin de lograr una (o más) de sus intenciones de
 % alcanzar una meta.
 % Los planes pueden ser complejos en mayor o menor medida, en función a
 % la cantidad de acciones que contiene.
 % En  particular, los planes pueden contener otros planes, dado que
 % satisfacer una meta puede requerir la satisfacción de metas
 % intermedias.
 % Esto refleja que en  el modelo de Bratman, inicialmente los planes son
 % concebidos sólo parcialmente,  y los detalles son incorporados a
 % medida que progresa su ejecución.


\section{Programaci�n L�gica Rebatible}

%\subsubsection{Representaci�n de conocimiento}

A continuaci�n introducimos las definiciones b�sicas necesarias para
representar conocimiento en Programaci�n L�gica Rebatible (\DLP). Para
un tratamiento exhaustivo, se remite al lector interesado al trabajo
de A. Garc�a y G. Simari\cite{delp04}.  En lo que sigue, se asume que
el lector posee un conocimiento b�sico acerca de los aspectos
fundamentales de la programaci�n l�gica.

\begin{definicion}(Programa \DLP\ \PP)

Un programa l�gico rebatible (delp) es un conjunto \PP\ = \SD\ donde
\SSet\ y \DD\ representan conjuntos de conocimiento \textit{estricto}
y \textit{rebatible}, respectivamente. El conjunto \SSet\ de
conocimiento estricto involucra \textit{reglas estrictas} de la forma
\srule{L}{Q_1,\ldots,Q_k} y  \textit{hechos} (reglas estrictas con
cuerpo vac�o), y se asume que es \textit{no-contradictorio}.  El
conjunto \DD\ de conocimiento rebatible involucra \textit{reglas
rebatibles} de la forma  \drule{L}{Q_1,\ldots,Q_k}, lo cual se
interpreta como ``$Q_1,\ldots,Q_k$ proveen razones tentativas  para
creer $L$''. Las reglas estrictas y rebatibles en \DLP\ son definidas
usando un conjunto  finito de literales. Un literal es un �tomo ($L$),
la negaci�n estricta de un �tomo ($\sim L$) o  la negaci�n
\textit{default} de un �tomo (\textit{not} $L$).

\end{definicion}

El lenguaje l�gico subyacente en \DLP\ es el de la programaci�n l�gica
extendida, enriquecido con el s�mbolo especial ``\drule{}{}'' para
denotar reglas rebatibles. Tanto la negaci�n  \textit{default} como la
cl�sica est�n permitidas (denotadas \textit{not} y \textit{$\sim$},
respectivamente). Sint�cticamente, el s�mbolo ``\drule{}{}'' es lo
�nico que distingue un regla \textit{rebatible}
\drule{L}{Q_1,\ldots,Q_k} de una regla \textit{estricta} (no-
rebatible) \srule{L}{Q_1,\ldots,Q_k}.  Las reglas \DLP\, por lo tanto,
son consideradas como \textit{reglas de inferencia} en lugar
implicaciones. De forma an�loga a la programaci�n l�gica tradicional,
la \textit{definici�n} de un predicado $P$ en \PP , denotado
$P^{\scriptsize{\PP}}$, est� dada por el conjunto de todas las reglas
(estrictas y rebatibles) con cabeza $P$  y aridad $n$ en \PP . Si $P$
es un predicado en \PP , entonces \textit{nombre(P)} y
\textit{aridad(P)} denotan el nombre y la aridad del predicado,
respectivamente. Escribiremos \textsf{Pred}(\PP) para denotar el
conjunto de todos los nombres de predicados definidos en un programa
\DLP\ \PP.

\subsection{Argumento, Contraargumento y Derrota}

Dado un programa \DLP\ \PP\ = \SD\, resolver consultas resulta en la
construcci�n de \textit{argumentos}. Un argumento \ArgA\ es un
conjunto (posiblemente vac�o) de reglas rebatibles fijas que junto al
conjunto \SSet\  provee una prueba l�gica para un dado literal \ArgQ,
satisfaciendo los requerimientos adicionales de  \textit{no-
contradicci�n} y \textit{minimalidad}. Formalmente:

\begin{definicion}[Argumento]

Dado un programa \DLP\ \PP, un argumento \ArgA\ para una consulta
\ArgQ, notado \AQ\, es un subconjunto de  instancias fijas de las
reglas rebatibles en \PP, tal que:
	
\begin{enumerate}[(1)]

\item existe una derivaci�n rebatible para \ArgQ de \SyA;

\item \SyA\ es no-contradictorio (\ie, \SyA\ no implica dos literales
complementarios $L$ y \lit{\no L} (o $L$ y \textsf{not}\ $L$), y,

\item \ArgA\ es minimal con respecto al conjunto inclusi�n (\ie, no
hay \Ap\ $\subset$ \ArgA\ tal que existe una derivaci�n rebatible para
\ArgQ\ de \SyAp).

\end{enumerate}
	
\end{definicion}

Un argumento \AaQa\ es un \textit{subargumento} de otro argumento
\AbQb\ si $\ArgAa \subseteq \ArgAb$. Dado un programa \DLP\ \PP,
\textit{Args(\PP)} denota el conjunto de todos los posibles argumentos
que  pueden ser derivados de \PP.

La noci�n de derivaci�n rebatible corresponde a la usual derivaci�n
SLD dirigida por consultas empleada en programaci�n l�gica, aplicando
\textit{backward chaining} a las reglas estrictas y rebatibles; en
este contexto, un literal negado \lit{\no P} es tratado simplemente
como un nuevo nombre de predicado \textit{no\_P}. La minimalidad
impone una especie de ``principio de la navaja de Occam'' sobre la
construcci�n  de argumentos. El requerimiento de no-contradicci�n
proh�be el uso de (instancias fijas de) reglas rebatibles en un
argumento \ArgA\ cuando \SyA\ deriva dos literales complementarios. Es
de notar que el concepto de no-contradicci�n captura los dos enfoques
usuales de negaci�n en la programaci�n l�gica (negaci�n
\textit{default} y negaci�n cl�sica), ambas presentes en \DLP\ y
relacionadas a la noci�n de contraargumento, como se muestra a
continuaci�n.

\begin{definicion}[\textbf{Contraargumento}]

Un argumento \AaQa\ es un \textit{contraargumento} para un argumento
\AbQb\ si y s�lo si

\begin{enumerate}[a)]

\item (ataque a subargumento) existe un subargumento \AQ\ de \AbQb\ 
(llamado \textit{subargumento en desacuerdo}) tal que el conjunto 
\SyQaQ\ es contradictorio, o

\item (ataque por negaci�n default) un literal \negda{\ArgQa}\ est�
presente en el cuerpo de alguna  regla en \ArgAb.

\end{enumerate}	
	
\end{definicion}

% La primer noci�n de ataque es tomada del framework de Simari-Loui; la
% �ltima est� relacionada al  enfoque argumentativo de programaci�n
% l�gica de Dung, asi como tambi�n a otras formalizaciones, como el
% trabajo de Prakken y Sartor, o el trabajo de Kowa y Toni.

Como en muchos marcos de argumentaci�n, vamos a asumir un
\textit{criterio de preferencia} para los  argumentos en conflicto
definido como la relaci�n $\preceq$, la cual es un subconjunto del
producto  cartesiano \textit{Args(\PP)} $\times$ \textit{Args(\PP)}.
Esto lleva a la noci�n de \textit{derrota} entre argumentos como una
refinaci�n del criterio de contraargumento. En particular, vamos a
distinguir entre  dos tipos de derrotadores, \textit{propios} y
\textit{por bloqueo}.

\begin{definicion}[\textbf{Derrotadores propios y por bloqueo}]

Un argumento \AaQa\ es un \textit{derrotador propio} para un argumento
\AbQb\ si \AaQa\ contra-argumenta \AbQb\ con un sub-argumento en
desacuerdo \AQ\ (ataque a subargumento) y \AaQa\ es estrictamente
preferido sobre \AQ\ con respecto a $\preceq$.

Un argumento \AaQa\ es un \textit{derrotador por bloqueo} para un
argumento \AbQb\ si \AaQa\ contra-argumenta \AbQb\ y una de las
siguientes situaciones se presenta: (a) Hay un sub-argumento en
desacuerdo \AQ\ para \AbQb, y \AaQa\ y \AQ\ no est�n relacionados
entre s� con respecto a $\preceq$; o (b) \AaQa\ es un ataque  por
negaci�n default sobre alg�n literal \negda{\ArgQa}\ en \AbQb.

El t�rmino \textit{derrotador} ser� usado para referirse
indistintamente a derrotadores propios o por bloqueo.

\end{definicion}

La especificidad generalizada es t�picamente usada como un criterio de
preferencia basado en la sintaxis  para argumentos en conflicto,
favoreciendo aquellos argumentos que est�n \textit{m�s informados} o
son  \textit{m�s directos}. A modo de ejemplo, consideremos tres
argumentos

\nlA{\AS{\drule{a}{b,c}}{a}}, \AS{\drule{\no $a$}{b}}{\no a} y
\AS{(\drule{a}{b});(\drule{b}{c})}{a} construidos sobre la base de un
programa \PP\ = \SD\ =
($\{b,c\},\{\drule{b}{c};\drule{a}{b};\drule{a}{b,c};\drule{\no
a}{b}\}$). Si se utiliza especificidad generalizada como criterio de
comparaci�n entre argumentos, el argumento  \AS{\drule{a}{b,c}}{a}
ser�a preferido sobre el argumento \AS{\drule{\no $a$}{b}}{\no a} ya
que el primero es considerado \textit{m�s informado} (\ie, est� basado
en m�s premisas). Sin embargo, el argumento  \AS{\drule{\no
$a$}{b}}{\no a} es preferido sobre
\AS{(\drule{a}{b});(\drule{b}{c})}{a} ya que el primero es considerado
\textit{m�s directo} (\ie, es obtenido a partir de una derivaci�n m�s
corta). Sin embargo, debe ser remarcado que, adem�s de especificidad,
otros criterios de preferencia alternativos pueden ser usados;  \eg,
aplicar prioridad sobre las reglas para definir la comparaci�n de
argumentos, o considerar valores  num�ricos correspondientes a medidas
asociadas a conclusiones de argumentos. El primer enfoque es empleado
en {\footnotesize D}-P{\footnotesize ROLOG}, L�gica Rebatible,
extensiones de la L�gica Rebatible, y programaci�n l�gica sin negaci�n
por falla. El segundo criterio fue el aplicado en el desarrollo del
sistema que se presenta en los cap�tulos siguientes.

\subsection{C�mputo de garant�as a trav�s de an�lisis dial�ctico}

Dado un argumento \AQ, pueden existir diferentes derrotadores $\BaQa$
$\ldots$ $\BkQk$, k $\ge$ 0 para \AQ. Si el argumento \AQ\ es
derrotado, entonces ya no estar�a soportando su conclusi�n \ArgQ.  Sin
embargo, dado que los derrotadores son argumentos, estos pueden a su
vez ser derrotados. Esto  induce un an�lisis dial�ctico recursivo
completo para determinar qu� argumentos son derrotados en  �ltima
instancia. Para caracterizar este proceso, primero se introducen
algunas nociones auxiliares.

Una \textit{l�nea argumentativa} comenzando en un argumento \AoQo\
(denotado $\lambda^{\scriptsize \AoQo}$) es una secuencia [\AoQo,
\AaQa, \AbQb,\ldots,\AnQn\ldots] que puede ser pensada como un
intercambio de  argumentos entre dos partes, un \textit{proponente}
(argumentos en posiciones pares) y un \textit{oponente} (argumentos en
posiciones impares). Cada \AiQi\ es un derrotador para el argumento
previo \AimQim\ en la secuencia, $i > 0$.

A fin de evitar razonamiento \textit{falaz} o mal-formado (\eg ,
lineas argumentativas infinitas), el  an�lisis dial�ctico impone
restricciones adicionales para que el intercambio de argumentos pueda
ser  considerado racionalmente \textit{aceptable}. Puede ser probado
que las l�neas argumentativas aceptables  son finitas. Un tratamiento
exhaustivo sobre restricciones de aceptabilidad pueden ser encontradas
en el trabajo de Garc�a y Simari\cite{delp04}.

%REF

Dado un programa \DLP\ \PP\ y un argumento inicial \AoQo, el conjunto
de todas las l�neas argumentativas aceptables comenzando en \AoQo\ da
lugar a un an�lisis dial�ctico completo para \AoQo\ (\ie, todos los
di�logos posibles sobre \AoQo\ entre proponente y oponente),
formalizado mediante un \textit{�rbol dial�ctico}.

Los nodos en un �rbol dial�ctico $T_{\scriptsize \AoQo}$ pueden ser
marcados como nodos \textit{derrotados}  y \textit{no derrotados}
(nodos D -\textit{defeated}- y nodos U -\textit{undefeated}-,
respectivamente).  Un �rbol dial�ctico ser� marcado como un �rbol
{\footnotesize AND-OR}: todas las hojas en  $T_{\scriptsize \AoQo}$
ser�n marcadas como nodos U (dado que no poseen derrotadores), y cada
nodo interno  ser� marcado como nodo D si y s�lo si tiene al menos un
nodo U como hijo, y como nodo U en otro caso.  Un argumento \AoQo\ es
finalmente aceptado como v�lido (o \textit{garantizado}) con respecto
a un programa  \DLP\ \PP\  si y s�lo si la ra�z del �rbol dial�ctico
asociado $T_{\scriptsize \AoQo}$ est� etiquetado como \textit{nodo U}.

%Super�ndice

Dado un programa \DLP\ \PP, resolver una consulta \ArgQ\ con respecto
a \PP\ implica determinar si \ArgQ\ est� soportado por (al menos) un
argumento garantizado. Diferentes actitudes dox�sticas %? pueden ser
distinguidas de la siguiente manera:

\begin{enumerate}[(1)]

\item textit{Yes}: se cree ArgQ si y s�lo si hay al menos un argumento
garantizado soportando ArgQ en base a PP.

\item \textit{No}: se cree \lit{\no \ArgQ} si y s�lo si hay al menos
un argumento garantizado  soportando \lit{\no \ArgQ} en base a \PP.

\item textit{Undecided}: ni ArgQ ni lit{no ArgQ} est�n garantizados
con respecto a PP.

\item textit{Unknown}: ArgQ no se encuentra en el lenguaje de PP.

\end{enumerate}


\section{Multi-Agent Programming Contest}
\label{sec:mapc}

El \textit{Multi-Agent Programming Contest} es un concurso de
programaci�n de Inteligencia Artificial iniciado en el a�o 2005 con el
objetivo de estimular la investigaci�n en el �rea de desarrollo y
programaci�n de Sistemas Multi-Agente. Para ello, la competencia
propone diferentes escenarios de juego de manera anual, que obligan a
los participantes tanto a identificar y resolver problemas clave, como
a explorar lenguajes, plataformas y herramientas de programaci�n para
Sistemas Multi-Agente.

\subsection{Escenario MAPC 2011}
\label{sec:escenario_mapc}

El escenario del a�o 2011 est� formado por el mapa de un planeta
representado mediante un grafo. Cada nodo del grafo es una locaci�n
v�lida (y tiene un valor determinado), y existen arcos (con diferente
costo de energ�a) que permiten a un agente desplazarse de una locaci�n
a otra.

En cada ronda de la competici�n participan dos equipos rivales. Cada
equipo posee un conjunto de agentes con diferentes roles
preestablecidos (\textit{Explorador}, \textit{Saboteador},
\textit{Reparador}, \textit{Sentinela} e \textit{Inspector}). El rol
de cada agente define tanto el conjunto de acciones que puede
realizar, como sus caracter�sticas f�sicas (\textit{Energ�a},
\textit{Salud}, \textit{Fuerza} y \textit{Rango de Visi�n}).

\subsubsection{Puntaje}
\label{sec:puntaje}

La simulaci�n del juego se desarrolla por pasos, y en cada paso se
otorga a los equipos una determinada cantidad de puntos seg�n el
estado de la simulaci�n. El objetivo del juego es obtener la mayor
cantidad de puntos posibles cuando la simulaci�n termina.

Para obtener puntos, los agentes de cada uno de los equipos deben
lograr formar \textit{"`zonas"'} en el mapa logrando posicionarse en
diferentes locaciones de manera estrat�gica. La predominancia de un
equipo sobre el otro en los nodos es determinada por un algoritmo bien
definido para la competencia, y el valor de todos los nodos dominados
por un equipo es el principal factor del puntaje otorgado en cada uno
de los pasos de la simulaci�n. Algunas otras situaciones, como el
logro de determinados \textit{achievements}, pueden otorgar puntos
adicionales y dinero al equipo.

\subsubsection{Acciones}
\label{sec:acciones}

Todos los agentes tienen acciones en com�n que pueden realizar en cada
uno de los pasos de la simulaci�n:

\begin{itemize}

\item goto(X): el agente se desplaza hacia el nodo X, siempre y cuando
exista un arco que conecte el nodo actual del agente con X, y dicho
arco tenga un costo menor a la energ�a actual del agente.

\item survey(X): el agente recibe en su pr�xima percepci�n los costos
de todos los arcos conectados al nodo en el que se encuentra
actualmente.

\item buy(X): el agente utiliza el dinero obtenido a partir de los
\textit{achievements} para aumentar el valor m�ximo de cualquiera de
sus caracter�sticas f�sicas (Energ�a, Salud, Fuerza o Rango de visi�n)
en 1 punto.

\item recharge: el agente recupera el 20% de su energ�a m�xima.

\item skip: el agente pasa al turno siguiente sin realizar ning�n tipo
de acci�n.

\end{itemize}

Adem�s, seg�n el rol de cada agente, existen algunas acciones
espec�ficas que pueden realizar:

\begin{itemize}

\item attack(X): acci�n disponible �nicamente para los
\textit{Saboteadores}; el agente ataca a un enemigo X, si dicho
enemigo se encuentra en el mismo nodo. El ataque, de tener �xito,
decrementa la energ�a del agente enemigo, pudiendo deshabilitarlo en
caso de que �sta llegue a 0.

\item parry: acci�n disponible �nicamente para los
\textit{Reparadores}, \textit{Saboteadores} y \textit{Sentinelas}. La
acci�n protege al agente de los ataques enemigos, impidiendo que �stos
tengan �xito.

\item probe: acci�n disponible �nicamente para los
\textit{Exploradores}. El agente recibe en su pr�xima percepci�n el
valor del nodo en el que se encuentra actualmente. �sta acci�n no s�lo
resulta importante por conocer el valor del nodo, sino que adem�s
permite que, cuando el nodo es conquistado por el equipo, dicho valor
se sume al total de puntos de la zona. Un nodo en el que no se realiz�
\textit{probe} suma �nicamente 1 punto al valor total de la zona.

\item inspect: acci�n disponible �nicamente para los
\textit{Inspectores}. El inspector recibe en su pr�xima percepci�n la
informaci�n f�sica (Salud, Energ�a, Fuerza, Rango de visi�n) de todos
los agentes enemigos que se encuentren en el mismo nodo que �l, o en
cualquier vecino directo.

\item repair(X): acci�n disponible �nicamente para los
\textit{Reparadores}. El reparador aumenta el valor de la Salud actual
de su compa�ero de equipo X (volviendo a habilitarlo, en caso de que
su Salud fuera 0).

\end{itemize}

\subsubsection{Toma de decisiones y motivaci�n para la resoluci�n de
conflictos}
\label{sec:toma_de_decisiones}

Dado que cada agente decide por separado qu� acci�n tomar, muchas
veces ocurre que dos (o m�s) de los agentes del mismo equipo realizan
acciones que resultan redundantes, peligrosas, y en el peor de los
casos, perjudiciales al combinarse. Como mencionamos en el ejemplo
introductorio de la tesis, en el caso de que dos agentes realicen una
acci�n id�ntica a la vez, existe la posibilidad de que dicha
planificaci�n represente un malgasto de tiempo o recursos para los
agentes; en el marco de MAPC 2011, esto ocurre, por ejemplo, cuando
dos agentes cualesquiera tienen como intenci�n
\textit{survey($X_{i}$)}, cuando dos exploradores tienen como
intenci�n \textit{probear($X_{i}$)}, o cuando dos inspectores tienen
como intenci�n \textit{inspect} estando en el mismo nodo.

Si bien por limitaciones temporales durante la competencia �nicamente
se realizaron coordinaciones impl�citas en las acciones de los
agentes, es naturalmente posible mejorar dicha coordinaci�n para sacar
mayor r�dito de las acciones, y �sta es la principal motivaci�n de
este trabajo.

