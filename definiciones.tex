% Contexto de la tesis (background formal, y contexto del desarrollo
\chapter{Definiciones preliminares} 

En este cap�tulo se revisar�n algunas definiciones de conceptos
t�cnicos, para posteriormente utilizarlos sin ambiguedad durante el
resto de la presentaci�n.

\section{Agente inteligente}

Un agente es una entidad computacional aut�noma, que puede percibir su
entorno a trav�s de sensores, y actuar en dicho entorno utilizando
efectores. Usualmente, la informaci�n que un agente percibe de su
entorno es s�lo parcial. Los agentes toman decisiones a partir de la
informaci�n contenida en su base de conocimiento, siguiendo diferentes
conjuntos de reglas propuestas, y act�an de manera acorde a la
decisi�n tomada. Dichas acciones, a su vez, pueden producir efectos en
el entorno.

Actualmente los agentes tienen un campo de aplicaci�n muy amplio y
existen muchos tipos de agentes diferentes (por ejemplo:
\textit{reactivos}, \textit{deliberativos}, \textit{inteligentes},
\textit{de interface}, \textit{colaborativos}), los cuales a su vez
est�n orientados a distintos entornos de aplicaci�n.

En la mayor�a de los casos, los agentes no existen por s� solos, sino
que participan de un Sistema Multi-Agente (SMA).

% El objetivo particular de este proyecto es la aplicaci�n de
% Argumentaci�n para la implementaci�n de di�logos entre agentes
% inmersos en un escenario con objetivos determinados. Puntualmente, se
% enfocar� la investigaci�n a la plataforma propuesta en el Multi-Agent
% Programming Contest, un juego acad�mico donde agentes independientes
% compiten por diferentes objetivos.\\ Sin embargo, el desarrollo de
% herramientas para implementar tales formalismos se encuentra en
% progreso y a un paso m�s lento. Adem�s, muchas de las herramientas
% disponibles carecen de una base formal y suelen ser simplemente un
% entorno de desarrollo amigable.\\

\section{Sistema Multi-Agente}

Es un sistema en el cual muchos agentes interact�an para conseguir
alg�n objetivo o realizar alguna tarea com�n. En los sistemas multi-
agente, cada agente tiene informaci�n incompleta y capacidades
limitadas, el control del sistema es distribuido, los datos est�n
descentralizados, y la computaci�n es asincr�nica. Los agentes se
desenvuelven en un entorno din�mico y cambiante, el cual no puede
predecirse y se ve afectado por las acciones que son llevadas a cabo.

Un aspecto importante en SMA es la comunicaci�n entre agentes, la cual
puede ser necesaria para que los agentes compitan o cooperen de
acuerdo a sus metas individuales. Los di�logos con otros agentes del
mismo ambiente son, actualmente, un �rea de estudio intensivo.

\input{preliminares_bdi.tex}

\input{preliminares_delp.tex}

\input{mapc.tex}
