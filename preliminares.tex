% Contexto de la tesis (background formal, y contexto del desarrollo

\chapter{Definiciones preliminares} 
\label{chap:definiciones_preliminares}

En este capítulo se revisarán algunas definiciones de conceptos
técnicos, para posteriormente utilizarlos sin ambigüedad durante el
resto de la presentación.

\section{Agente inteligente}
 \label{sec:agente_inteligente}
 
 % Un agente es una entidad computacional autónoma, que puede percibir su
 % entorno a través de sensores, y actuar en dicho entorno utilizando
 % efectores.
 % Usualmente, la información que un agente percibe de su entorno es sólo
 % parcial.
 % Los agentes toman decisiones a partir de la información contenida en
 % su base de conocimiento, siguiendo diferentes conjuntos de reglas
 % propuestas, y actúan de manera acorde a la decisión tomada.
 % Dichas acciones, a su vez, pueden producir efectos en el entorno.
 
 % Actualmente los agentes tienen un campo de aplicación muy amplio y
 % existen muchos tipos de agentes diferentes (por ejemplo:
 % \textit{reactivos}, \textit{deliberativos}, \textit{inteligentes},
 % \textit{de interface}, \textit{colaborativos}), los cuales a su vez
 % están orientados a distintos entornos de aplicación.
 
 % En la mayoría de los casos, los agentes no existen por sí solos, sino
 % que participan de un Sistema Multi-Agente (SMA).

\section{Sistema Multi-Agente}
 \label{sec:sistema_multiagente}
 
 % En un Sistema Multi-Agente (SMA) mas de un agente interactúan para
 % lograr un objetivo o realizar una tarea común.
 % Cada agente tiene información incompleta y capacidades limitadas, el
 % control del sistema es distribuido, los datos están descentralizados,
 % y la computación es asincrónica.
 % Los agentes se desenvuelven en un entorno dinámico y cambiante, el
 % cual no puede predecirse y se ve afectado por las acciones que son
 % llevadas a cabo.
 
 % Un aspecto importante en SMA es la comunicación entre agentes, la cual
 % puede ser necesaria para que los agentes compitan o cooperen de
 % acuerdo a sus metas individuales. 
 % Los diálogos con otros agentes del mismo ambiente son, actualmente, un
 % área de estudio intensivo.
