
\chapter{Servidor de Percepciones}
 \label{chap:servidor_de_percepciones}

    The PS maintains a connection for each agent.  The connection handling
    methods encode the associate array into a form suitable for conversion into
    a set datastructure, which is then sent over the network.  On each
    iteration, the PS waits for each agent's data, performs a union of all data
    sets, and returns to each agent the set difference between the data the
    agent sent and the total union.  Figures \ref{fig:pythonperceptpublic} and
    \ref{fig:pythonperceptprivate} show example percepts after parsing, before
    being sent to the percept server.

    \begin{figure}
    \centering
    
    \begin{small}
    \begin{verbatim}
    { 'surveyed_edges' : [ ], 
      'vis_verts'      : [ { 'name': 'vertex65',  
                             'team': 'none' }, 
                           ...  ],  
      'vis_ents'       : [ { 'node':   'vertex97',  
                             'status': 'normal', 
                             'name':   'a6',  
                             'team':   'A' }, 
                             ...  ],  
      'inspected_ents' : [ ],  
      'vis_edges'      : [ { 'node1': 'vertex141', 
                             'node2': 'vertex65' }, 
                           ...  ], 
      'position'       : [ { 'node':       'vertex141', 
                             'vis_range':  '1', 
                             'health':     '6', 
                             'name':       'self', 
                             'max_health': '6' } ], 
      'probed_verts'   : [ ] }
    \end{verbatim}
    \end{small}
    \caption{A sample public section of a percept, after parsing.}
    \label{fig:pythonperceptpublic}
    \end{figure}

    \begin{figure}
   


    \begin{verbatim}
    { 'total_time':      2000L, 'zone_score': '0',            
      'last_step_score': '20',  'timestamp':  '1323732915832',       
      'strength':        '0',   'energy':     '11',                  
      'money':           '12',  'max_energy_disabled': '16',                  
      'last_action':     'recharge', 'max_health':     '6',
      ... }
    \end{verbatim}
    \caption{A sample private section of a percept.}
    \label{fig:pythonperceptprivate}
    \end{figure}

%--------------------------------------------CONEXION PERCEPT-SERVER-%
\section[Conexión con el Percept Server]
 {Conexión del Agente con el Servidor de Percepciones}
 \label{sec:conexion}

 TODO

%---------------------------------------------------FORMATO PERCEPTS-%
\section{Formato de mensajes}
 \label{sec:formato_mensajes}

 TODO

%-------------------------------------------PROTOCOLO PERCEPT-SERVER-%
\section[Protocolo del Percept Server]
 {Protocolo de Comunicación con el Servidor de Percepciones}
 \label{sec:protocolo_perceptserver}

 TODO

\subsection{Esquema de Codificación}
 \label{sub:esquema_codificacion}

 Tras la recepción de la percepción del servidor, el parseo del XML y
 su transformación en diccionarios nativos de Python, el agente
 retransmite la informacion recibida al servidor de percepciones.
 Aunque es factible enviar la percepcón en formato XML o como la
 representación de la estructura de datos en memoria del agente, unos
 de los objetivos es distribuir tanto como posible el trabajo realizado
 por el sistema.
 Como las tareas principales realizadas por el servidor de percepciones 
 son la union de la información y la diferencia de cada percepción 
 individual con el total, se implemento un esquema más de codificacion. 
 Se transforma el diccionario que contiene la percepción en una 
 estructura sobre la cual se pueden realizar operaciones de conjunto de 
 manera eficiente. 

 TODO

\subsection{Esquema de Reconexión para Agentes Caidos}
 \label{sub:esquema_reconexion}
 
 Si el agente pierde la conexión, 
 
\subsection{Evaluación de Performance}
 \label{sub:evaluacion_performance}
