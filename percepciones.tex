
\chapter{Servidor de Percepciones}
\label{chap:servidor_de_percepciones}

\begin{verbatim}
[ ] capa de conexion con el servidor massim
[ ] formato de los mensajes percepciones/acciones
[ ] esquema de codificacion de las percepciones para mandar al PS (estructura y cpickle)
[ ] protocolo de comunicacion con el PS
[ ] esquema de reconexion para agentes caidos
[ ] algoritmo del PS
\end{verbatim}

\section{Conexión del Agente con el Servidor MASSim}

El protocolo de comunicación con el servidor MASSim especificado por 
el enunciado del concurso MAPC define que los agentes y el servidor 
intercambian mensajes en formato XML codificados por UTF-8, con un 
byte nulo para indicar el final del mensaje. 

\subsection{Protocolo de comunicación con el servidor MASSim}

TODO

\subsection{Formato de mensajes}

TODO

\subsubsection{Percepciones}

TODO

\subsubsection{Acciones}

  Las acciones se representan con la cadena de caracteres:
  
  \begin{verbatim}
  <?xml version="1.0" encoding="UTF-8" standalone="no"?>
    <message type="action">
      <action id=\"ID\" type=\"TYPE\"/>
    </message>\0
  \end{verbatim}
  
  En el caso de las acciones que requieren un parámetro adicional, la
  representación es:
  
  \begin{verbatim}
  <?xml version="1.0" encoding="UTF-8" standalone="no"?>
    <message type="action">
      <action id="ID" param="PARAM" type="TYPE"/>
    </message>\0
  \end{verbatim}
  
  donde {\tt ID} es el identificador de mensaje enviado por el servidor
  MASSim en la percepcion, {\tt TYPE} es el tipo de accion y {\tt PARAM}
  es el parametro. 
  
  {\tt TYPE} puede ser uno de:
  
  \begin{itemize}
  \item \tt{skip}
  \item \tt{goto}
  \item \tt{attack}
  \item \tt{parry}
  \item \tt{probe}
  \item \tt{survey}
  \item \tt{inspect}
  \item \tt{repair}
  \item \tt{recharge}
  \item \tt{buy}
  \end{itemize}

\section{Conexión del Agente con el Servidor de Percepciones}

TODO

\section{Formato de Percepciones}

TODO

\section{Protocolo de Comunicación con el Servidor de Percepciones}

TODO

\section{Esquema de Codificación}

 Tras la recepción de la percepción del servidor, el parseo del XML y
 su transformación en diccionarios nativos de Python, el agente
 retransmite la informacion recibida al servidor de percepciones.
 Aunque es factible enviar la percepcón en formato XML o como la
 representación de la estructura de datos en memoria del agente, unos
 de los objetivos es distribuir tanto como posible el trabajo realizado
 por el sistema.
 Como las tareas principales realizadas por el servidor de percepciones 
 son la union de la información y la diferencia de cada percepción 
 individual con el total, se implemento un esquema más de codificacion. 
 Se transforma el diccionario que contiene la percepción en una 
 estructura sobre la cual se pueden realizar operaciones de conjunto de 
 manera eficiente. 
 
TODO

\subsection{Esquema de Reconexión para Agentes Caidos}

Si el agente pierde la conexión, o 
