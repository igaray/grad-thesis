\subsubsection{Base de conocimiento}
 \label{subsub:base_de_conocimiento}

 % Como fue mencionado, la percepción del agente en cada iteración es
 % convertida a una estructura de datos que permite, de manera más
 % sencilla, manipular y compartir la información.
 % Cuando el agente cuenta con todos los datos relativos a las
 % percepciones del equipo, la base de conocimiento puede ser
 % actualizada convenientemente.
 % Una colección de predicados de Prolog consultados desde el programa
 % principal se encarga de verificar que la información existente no
 % resulte sobreescrita, y que información redundante no sea incorporada.

 % La información que constituye conocimiento certero sobre el estado del
 % escenario es almacenada mediante términos, que sirven como parámetros
 % del predicado \texttt{k/1} (\textit{knowledge}).
 % Cada uno de los datos de interés es representado mediante un término
 % diferente.
 % En muchos casos, esta clase de términos incluyen un parámetro ligado
 % al número de turno en el cual el dato fue percibido.
 % De esta forma, es posible realizar ciertos análisis, como por
 % ejemplo, considerar obsoleta la información de una determinada
 % antigüedad.

 % \begin{verbatim}
 %   k(equipoAgente(Agente, Equipo)).
 %   k(valorNodo(Nodo, Valor)).
 %   k(arco(Nodo1, Nodo2, Costo)).
 %   k(posicionAgente(Agente, Turno, Posición)).
 %   k(equipoNodo(Turno, Nodo, Dueño)).
 % \end{verbatim}

 % Las creencias que provienen de inferencias y cálculos realizados a
 % partir de información ya existente también son almacenadas mediante
 % términos, en este caso parámetros %argumentos del predicado
 % \texttt{b/1} (\textit{beliefs}).

 % Este tipo de creencias es empleado directamente por el módulo
 % encargado de la toma de decisiones, y se mantienen vigentes sólo
 % durante el turno en el cual fueron generadas. Es decir, que, al
 % finalizar cada turno, son descartadas para evitar futuros problemas o
 % inconcistencias.

 % \begin{verbatim}
 %   b(estoyEnLaFrontera).
 %   b(posibleExplorar(Nodo)).
 %   b(haySaboteador(Nodo)).
 % \end{verbatim}

 % Existe cierta información que es formulada de manera hipotética.
 % Se trata de datos surgidos de suposiciones realizadas sobre posibles
 % estados futuros del escenario, a partir de su estado actual.
 % Este tipo de datos resulta fundamental para facilitar los cálculos
 % realizados por los algoritmos que se encargan de buscar formas de
 % maximizar el puntaje del equipo.
 % Dado que no constituye información real, sino posible a futuro, se
 % almacena mediante parámetros de un predicado especial, \texttt{h/1}
 % (\textit{hypothetical}).

 % \begin{verbatim}
 %   h(nodoEquipo(Nodo, Dueño)).
 %   h(posicion(Turno, Agente, Nodo)).
 % \end{verbatim}

 % Las intenciones surgen del proceso argumentativo explicado más
 % adelante, y son representadas utilizando términos.
 % Si la intención no posee argumentos, entonces es representada mediante
 % un átomo.
 % En otro caso, se emplea un functor que denota el nombre de la
 % intención, acompañado por un argumento.
 % Las acciones, por el contrario, son representadas a través de listas.
 % El primer elemento de la lista es un atómo denotando el tipo de
 % acción.
 % El resto de la lista contiene, ocasionalmente, un término que indica
 % el argumento de la acción, como por ejemplo el nombre de un nodo o un
 % agente.
 % Los planes son representados mediante listas de acciones, es decir,
 % listas de listas.

 % Contrario a lo que ocurre con las creencias, tanto las intenciones
 % como los planes constituyen información que debe perdurar en la base
 % de conocimiento tantos turnos como sea necesario.
 % Para este tipo de datos se emplean hechos específicos que cuentan con
 % un único argumento.

 % \begin{verbatim}
 %   intencion(explorar(vertex7)).
 %   plan([[recharge], [goto, vertex7], [survey]]).
 % \end{verbatim}

%------------------------------------------------------PLANIFICACION-%
\subsection{Planificación}
 \label{sub:planificacion}

 % La planificación consiste en obtener la secuencia de acciones que
 % llevan al cumplimiento de la intención propuesta.
 % Esta lista está compuesta por las acciones que le permiten al agente
 % posicionarse en el nodo deseado, y, en algunos casos, una acción
 % concreta a realizar. Como se dijo anteriormente, en la etapa de seteo
 % de creencias, todos los caminos hallados por el algoritmo de búsqueda
 % son almacenados. Dicho algoritmo fue implementado de manera tal que
 % los caminos no están constituidos por nodos o vértices, sino por una
 % secuencia optimal de acciones, que tiene en cuenta no sólo el nodo
 % destino, sino también los recursos del agente, y la meta final a
 % realizar (en caso de haber una acción final).
 % De esta forma, cualquiera haya sido la intención elegida, el agente
 % cuenta en su base de conocimiento con el plan necesario para
 % cumplirla.
 % La planificación se resume entonces a tomar las acciones
 % correspondientes, y establecerlas efectivamente como el plan a seguir.

 % Alternativamente, esta etapa puede introducir ciertas acciones con el
 % objetivo de optimizar el uso del turno.
 % En aquellas situaciones en que el agente se dispone a permanecer
 % inactivo, la acción nula (\texttt{skip}) puede ser reemplazada por la
 % acción de recargar energía, si es que esta resulta más productiva.

%----------------------------------------------------------EJECUCION-%
\subsection{Ejecución}
 \label{sub:ejecucion}

 % Dado que el plan se encuentra almacenado de manera completa y
 % ordenada, la ejecución se realiza en forma directa.
 % Se toma la próxima acción, es decir, la primera acción del plan
 % restante, y se la retorna al módulo principal del programa.
 % Éste se encarga posteriormente de enviarla al entorno, para que se
 % convierta finalmente en la siguiente acción realizada por el agente.

\subsubsection{Condición de corte}
 \label{subsub:condicion_de_corte}

 % Existen situaciones en las que el paso de los turnos genera que el
 % cumplimiento de una meta se vuelva inalcanzable, innecesario,
 % riesgoso, o menos productivo de lo previsto, por lo que resulta más
 % beneficioso abortar el plan existente, y seleccionar una nueva
 % intención.
 % Ésta es una etapa de verificación, que tiene como objetivo la
 % detección de este tipo de situaciones.
 % Es ejecutada sólo en aquellos turnos en los que el agente se encuentra
 % siguiendo el plan de una intención previamente determinada.

 % Cada deseo o esquema de deseo cuenta con una serie de
 % \textbf{condiciones de corte}, que son evaluadas al inicio de cada
 % turno, en caso de existir un plan establecido.
 % Si se verifica que alguna de estas condiciones se satisface, entonces
 % la intención es descartada, y el agente ingresa en un nuevo proceso de
 % selección. Entre las condiciones de corte tenidas en cuenta, se
 % encuentran:

 % \begin{itemize}
 % \item Que haya pasado una determinada cantidad de turnos desde el
 % inicio del plan.

 % \item Que el agente se encuentre deshabilitado.

 % \item Que haya sido atacado o se encuentre amenazado por un enemigo.

 % \item Que la meta haya sido alcanzada por un compañero de equipo.
 % \end{itemize}

\subsubsection{Re-planificación}
 \label{subsub:replanificacion}

 % La fase de re-planificación consiste en elaborar nuevamente el plan
 % que permite alcanzar la meta propuesta, sin modificar dicha meta. Este
 % paso, como el anterior, se realiza en los turnos en los que el agente
 % posee un plan pre-calculado.
 % Dado que en estos turnos no es necesaria la obtención de una nueva
 % intención, proceso que implica el mayor insumo de tiempo, la inclusión
 % de la re-planificación no afecta el funcionamiento normal del agente,
 % en términos de tiempo de ejecución.

 % Por el contrario, existe una mejora en el desempeño del equipo,
 % surgida de un mejor aprovechamiento de la información percibida.
 % Los agentes actualizan su información sobre el estado del mundo en
 % cada turno.
 % Datos como el estado en que se hallan los recursos del agente, la
 % incorporación de nodos y arcos hasta el momento desconocidos, o las
 % nuevas ubicaciones de los otros agentes, permiten elaborar planes más
 % precisos y ajustados a la realidad que los originalmente diseñados.
 % Así, los agentes son capaces de cumplir sus metas con mayor facilidad,
 % o abortarlas si es necesario.

